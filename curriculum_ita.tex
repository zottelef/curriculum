%% xelatex --interaction=nonstopmode curriculum_ita

\documentclass{curriculum}

\addbibresource{curriculum.bib}

\usepackage[italian]{babel}
\usepackage{fancyhdr}
\pagestyle{fancy}
\setlength{\headheight}{14pt}
\renewcommand{\headrulewidth}{0pt}
\renewcommand{\footrulewidth}{0pt}
\cfoot{\footnotesize \addfontfeature{Color=lightgray} Autorizzo il trattamento dei miei dati personali ai sensi del Dlgs 196 del 30 giugno 2003 e dell’art. 13 GDPR (Regolamento UE 2016/679) ai fini della ricerca e selezione del personale}

\renewcommand*{\mkbibnamegiven}[1]{%
  \ifitemannotation{highlight}
    {\textbf{#1}}
    {#1}}

\renewcommand*{\mkbibnamefamily}[1]{%
  \ifitemannotation{highlight}
    {\textbf{#1}}
    {#1}}

\begin{document}
  \header{Fabio}{Zottele} {tecnologo sperimentatore di quarto livello}
  
    Mi sono laureato in ingegneria nel 2005 e nella prima parte del mio percorso professionale ho acquisito le competenze per realizzare mappe digitali e per pubblicarle online.\\
    Dal 2006 lavoro presso la Fondazione Edmund Mach di San Michele all'Adige: attualmente sono assegnato al Centro di Trasferimento Tecnologico, Dipartimento Ambiente e agricoltura di montagna, Unità Agrometeorologia e sistemi informatici.
    
    Opero in autonomia negli ambiti:
    \begin{description}[style= unboxed, leftmargin= 6 pt, topsep= -3 pt, parsep= 3 pt, itemsep= 2pt]
     \item[Analisi di dati per il mio gruppo di lavoro.] Utilizzo metodi statistici per l'interpolazione spaziotemporale delle misure effettuate dai sensori dalla nostra rete agrometeorologica.\\
     L'obiettivo è quello di controllare, correggere e validare le misure e utilizzare le informazioni ottenute sia per realizzare mappe e analisi climatiche come il calcolo del rischio e del tempo di ritorno di eventi meteorologici che possono compromettere la produzione agricola\\
     Inoltre, sviluppo software per stimare il fabbisogno irriguo delle principali colture trentine in base all'evoluzione delle condizioni atmosferiche, della fisiologia della pianta e dello stato del terreno.
     %\\In questo ambito di lavoro ho supervisionato due collaboratoria progetto, un tirocinante e un tesista.    
     \item[Analisi di dati a supporto di altri gruppi della Fondazione Mach.] Metto a disposizione le mie conoscenze matematiche e statistiche in maniera trasversale. Individuo gli strumenti più adatti per analizzare serie di dati agronomici e ambientali e per la successiva interpretazione dei risultati.
     %Nell'ambito di questa attività ho supervisionato due tesisti.
     \item[Studio dell'evoluzione dei paesaggi viticoli trentini] Dal 2010 indago come l'interazione tra clima, economia e società guidi l'evoluzione del paesaggio viticolo montano.\\
     In origine, questo ambito di ricerca analizzava gli elementi fisici del paesaggio per confrontare differenti territori viticoli europei (Val di Cembra in Italia e  Banyuls\textendash sur\textendash mer in Francia).\\
     Successivamente, per integrare "il fattore umano" nella descrizione dell'evoluzione di un territorio nello spazio e nel tempo, ho utilizzato la \textit{teoria dei sistemi ad agenti multipli}: dalla sintesi di queste esperienze ho definito il concetto  di "Capitale Paesaggistico", \textit{Landscapital}, ovvero uno strumento che consente di dare un valore tangibile al paesaggio e per valorizzazione le produzioni vitivinicole e per confrontare paesaggi viticoli con caratteristiche differenti. 
     %Nell'ambito di questa attività ho coordinato uno studente di master ed un tesista.
     \item[Droni e agricoltura di precisione] A partire dall'esperienza maturata sviluppando sistemi di supporto alle decisioni in ambito agronomico, ho recentemente approfondito le opportunità ed i benefici che meccatronica può offrire all'agricoltura trentina.\\
     In particolare, ho coordinato un progetto per l'utilizzo del drone come strumento economicamente vantaggioso per il studiare l'evoluzione del paesaggio viticolo in forte pendenza. Inoltre, coordino un progetto per testare se il drone possa coadiuvare il monitoraggio dei giallumi della vite.\\
     Sono referente per la Fondazione Mach sul Tavolo della Ricerca e della Sperimentazione, nato a seguito della firma del "Protocollo di intesa per lo sviluppo della formazione aeronautica in Trentino".
     %Nell'ambito di questa attività ho coordinato un tirocinante ed due studenti: uno dell'Alta Formazione per Tecnico Superiore del Verde ed uno studente dell'Istituto Agrario.
    \end{description}
    \vspace{3pt}
    Nell'ambito di queste attività ho coordinato il lavoro di due collaboratori a progetto laureati in ingegneria, due studenti di master, tre tesisti universitari, un tirocinante laureato, due tirocinanti diplomati e tre studenti di scuola superiore.
    
    \newpage
    
    \section{Istruzione}
    \begin{entrylist}
    \entry
        {1995--2005}{M.Sc. magna cum laude {\normalfont In Ingegneria per l'Ambiente ed il Territorio}}{Università degli studi di Trento}{\emph{Un approccio 3D vettoriale alla modellazione di venti locali termicamente forzati.}\hfill\vspace{2pt}\\
        Partendo da un modello micrometeorologico, ho sviluppato un software per il GIS GRASS per descrivere il profilo di vento e di temperatura sulle pendici delle valli di tipo alpino.}
    \end{entrylist}
    
    
    \section{Esperienza}
    \begin{entrylist}
    \entry 
        {2006--Oggi}{Fondazione Edmund Mach} {San Michele all'Adige, IT}{\emph{Contrattista a progetto e tecnologo di IV livello}\hfill\vspace{3pt}\\
        Progetto e realizzo strumenti a supporto della gestione dell'acqua per l'irrigazione dei vigneti e dei frutteti trentini sviluppando software che stimano nello spazio e nel tempo il contenuto d'acqua nel suolo.
        \\Negli anni ho approfondito le tecniche di analisi statistica e geostatistica nell'ambito della meteorologia agraria. Ho sviluppato numerosi webGIS e, più recentemente, \textit{Rich Internet Applications} per la pubblicazione online delle elaborazioni effettuate nell'ambito delle attività del mio gruppo di lavoro. 
        \\Queste competenze mi hanno permesso di collaborare con numerosi ricercatori e tecnologi della Fondazione Mach su numerosi progetti ed attività: per prevedere l'impatto del cambiamento climatico sulla fenologia delle piante e sulla diffusione di patologie nelle foreste delle Alpi, per lo sviluppo di modelli diffusione spazio-temporale del danno causato da specie aliene di insetti nei vigneti del Trentino al fine di suggerire interventi agronomici mirati e ridurre conseguentemente l'utilizzo di biocidi, sull'elaborazione dei dati meteorologici per progettazione energetica in edilizia, e per utilizzare le immagini radiometriche satellitari per pianificare e gestire operazioni di soccorso in caso di calamità\ldots  
        \\Sono stato co-autore e docente al \textit{Corso di Sistemi e Tecnologie Informative Geografiche} per il Corso di Alta formazione per Tecnico del Verde, al corso \textit{Programmare con R} per l'associazione Italiana di Agrometeorologia, ai due seminari \textit{Workshop di R} per l'Unità di ricerca per il monitoraggio e la pianificazione forestale di CREA; al corso \textit{R: Spatial data structures and geostatistics} per Ricerca sul Sistema Energetico (RSE SpA); ed al corso \textit{Generalità sul clima; Microclima e Macroclima; elementi climatici: luce, temperatura, idrometeore, vento; strumenti di rilevazione e valutazione degli elementi climatici; Banche dati: rilevazioni e informazioni georeferenziate} per il Collegio dei Periti Agrari e dei Periti agrari Laureati.}
    \end{entrylist}
    
    \begin{entrylist}
    \entry
        {2005--2015}{Dipartimento di Ingegneria Civile e Ambientale dell'Università degli Studi di Trento}{Centro per la Difesa idrogeologica dell'Ambiente Montano}{\emph{Collaboratore a progetto per l'attività di ricerca e per corsi di formazione}\hfill\vspace{3pt}\\
        Durante il biennio 2005--2006 ho collaborato allo sviluppo di webGIS per il monitoraggio del traffico e sulla modellazione degli inquinanti nell'area urbana di Trento; per la pubblicazione online del piano di smaltimento dei rifiuti della regione Lazio; per la disseminazione dei dati geografici per le attività didattiche della Facoltà di Ingegneria dell'Università degli Studi di Trento; per un sistema di coordinamento delle operazioni di recupero dispersi in ambiente montano(SAR); e per l'integrazione di differenti basi dati per la gestione dei servizi urbani in collaborazione con INSULA SpA, braccio operativo del Comune di Venezia per la manutenzione urbana, le infrastrutture e l'edilizia.
        \\In questo periodo ho acquisito competenze nella progettazione e nella realizzazione di database relazionali con estensione geografica verificandone l'interoperabilità con i principali software GIS liberi e a licenza proprietaria.
        \\Fino al 2005 ho collaborato come co-autore, docente ed esercitatore al corso \textit{Grass Free ed Open Source GIS e geodatabase teoria ed applicazioni} (dalla terza alla dodicesima edizione) presso l'Università di Trento, l'Università di Genova. Al FOSS42006 di Losanna (CH) sono stato co-autore ed esercitatore del workshop \textit{GRASS GIS and external RDBMS}. 
        Nel 2006 sono stato esercitatore del corso \textit{Ortofotocarte da immagini satellitari ad alta risoluzione: metodologie, applicazioni e problemi} dell'Università di Roma, La Sapienza.}
    \end{entrylist}
    
    \section{Capacità e competenze}
        Realizzo codici informatici in numerosi linguaggi di programmazione e di scripting sia \textit{general purpose} (C), sia orientato a:
        \begin{description}[labelindent= 1em, labelsep*=0.5em, leftmargin=!, style= standard, font=\normalfont\textit]
            \item[statistica:] R, Stan, Julia;
            \item[web:] php, CSS, HTML, Mapscript e JavaScript (ECMA-262);
            \item[database:] SQL:2011;
            \item[testi:] \hologo{LaTeX}, Markdown;
            \item[sistemi complessi:] Netlogo, GAMA.
        \end{description}
        %Per lo sviluppo di codice orientato alla statistica utilizzo R, Stan, e Julia; per quello \textit{general purpouse} ho utilizzo C; per la programmazione  \textit{web oriented} utlizzo php, CSS, HTML, Mapascript e javaScript (ES5); per la gestione dei database utilizzo SQL; per la scrittura e la creazione automatica di testi utilizzo \LaTeX~e (R)Markdown; e per la modellazione multiagente utilizzo NetLogo e GAMA.
        
        Ho una conoscenza approfondita dei Sistemi e delle tecnologie Informative Geografiche (Qgis, GRASS, PostGIS e Mapserver) e dei database relazionali (PostgreSQL e MySQL). 
        
        Il coordinato il primo seminario "L'agroecologia del vigneto" tenutosi il 16 novembre 2017 a San Michele all'Adige nell'ambito degli eventi sul territorio del triveneto "Per una Viticoltura agroecologica" e la giornata tecnica "Droni e agricoltura: esperienze formative e applicazioni tecniche" tenutosi a San Michele all'Adige il 25 gennaio 2018.
        
        Ho partecipato alla stesura di numerosi progetti europei di cui  LIFE 2015, un Alpine Space, un Central Europe, un PRIMA-MED, un "L.P. 6/99" ed un PEI mantenedo una rete di contatti con enti locali, nazionali e d internazionali che comprende la Provincia Autonoma di Trento, HIT Trentino, Trentino Sviluppo, STEP Fraunhofer Italia Research, EIT Digital, Università di Roma la Sapienza, Università di Limoges, CREA, CIRAD, CERVIM, ITLA, ItalFly, Optoi Microelectronics, EOPTIS, BTLabs Russia, Termodrone, Università di Gaisenheim ed Università de La Laguna.
        
        Sono in possesso dell'attestato di pilota APR (I.APRA.006010) per la classe VL/MC conseguito in data 30 novembre 2017.
        
        Sono rappresentante sindacale dei lavoratori della Fondazione Edmund Mach iscritti alla Confederazione Generale Italiana del Lavoro (CGIL), membro della Free Software Foundation, sono Sommelier e socio dell'Associazione Italiana Sommelier (AIS) e socio dell'Associazione Nazionale Assaggiatori Grappa (ANAG).
        
        Dal 2005 al 2017 sono stato iscritto all'Ordine degli Ingegneri della Provincia di Trento, Ing. civile e ambientale, industriale e dell'informazione. Iscr. Albo n. 2912 - Sezione A degli Ingegneri. 
        
        \begin{table}[h]
        \centering
        \fontsize{9}{10}\selectfont
            \begin{tabular}{l c c c l c c c l c c c l c c c l c c c l}
                \toprule
                & & \multicolumn{7}{c}{\textbf{Comprensione}} & & \multicolumn{7}{c}{\textbf{Parlato}} & & \multicolumn{3}{c}{\textbf{Scritto}}\\
                & & \multicolumn{3}{c}{Ascolto} & &\multicolumn{3}{c}{Lettura} & & \multicolumn{3}{c}{Interazione} & &\multicolumn{3}{c}{Produzione orale}  & & \multicolumn{3}{c}{}\\
                \cmidrule{3-9} \cmidrule{11-17} \cmidrule{19-21}
                \textbf{Inglese} & \phantom{abc} & C1 & \phantom{a} & Livello avanzato & \phantom{ab} & C2 & \phantom{a} & Livello avanzato & \phantom{ab} & B2 & \phantom{a} & Livello intermedio & \phantom{ab} & C1 & \phantom{a} & Livello avanzato & \phantom{ab} & C2 & \phantom{a} & Livello avanzato\\
                \textbf{Francese} & \phantom{abc} & A2 & \phantom{a} & Livello elementare & \phantom{ab} & A2 & \phantom{a} & Livello elementare & \phantom{ab} & A2 & \phantom{a} & Livello elementare & \phantom{ab} & A2 & \phantom{a} & Livello elementare & \phantom{ab} & A2 & \phantom{a} & Livello elementare\\
                \textbf{Tedesco}  & \phantom{abc} & A2 & \phantom{a} & Livello elementare & \phantom{ab} & A2 & \phantom{a} & Livello elementare & \phantom{ab} & A2 & \phantom{a} & Livello elementare & \phantom{ab} & A2 & \phantom{a} & Livello elementare & \phantom{ab} & A2 & \phantom{a} & Livello elementare\\
                \bottomrule
            \end{tabular}
            \caption*{\addfontfeature{Color=lightgray}\small Autovalutazione delle competenze linguistiche secondo il "Quadro europeo di riferimento per le lingue"}
        \end{table}

    \section{Corsi e certificazioni}
    
    \begin{entrylist}
    \entry
        {2018}{Fondazione Edmund Mach}{Ciaotech}{\emph{Corso di europrogettazione}}
    \end{entrylist}
    
    \begin{entrylist}
    \entry
        {2018}{Fondazione Edmund Mach}{-}{\emph{Introduzione all'utilizzo consapevole del sistema di calcolo di KORE di FBK}}
    \end{entrylist}
    
    \begin{entrylist}
    \entry
        {2018}{Fondazione Edmund Mach}{-}{\emph{Fotogrammetria aerea, rilievi topografici ed architettonici con drone}}
    \end{entrylist}
   
    \begin{entrylist}
    \entry
        {2018}{Fondazione Edmund Mach}{ItalFLY}{\emph{Corso conseguimento attestato APR}}
    \end{entrylist}
    
    \begin{entrylist}
    \entry
        {2017}{GiPro giovani e professionisti}{Ordine Avvocati di Rovererto}{\emph{Corso Teorico-pratico per l'Accesso all'Amministrazione di Società: approfondimento in bilancio, pianificazione, businness model e change management}}
    \end{entrylist} 
    
    \begin{entrylist}
    \entry
        {2016}{Coursera}{University of Washington}{\emph{Machine Learning Foundations: A Case Study Approach}}
    \end{entrylist}
    
    \begin{entrylist}
    \entry
        {2015}{Coursera}{Duke University}{\emph{Data Analysis and Statistical Inference}}
    \end{entrylist}
    
    \begin{entrylist}
    \entry
        {2015}{Fondazione Edmund Mach}{-}{\emph{Diffusione del modello di organizzazione e di gestione (MOG)}}
    \end{entrylist}
    
    \begin{entrylist}
    \entry
        {2015}{Coursera}{Tel Aviv University}{\emph{What a Plant Knows (and other things you didn’t know about plants)}}
    \end{entrylist}

    \begin{entrylist}
    \entry
        {2015}{Fondazione Edmund Mach}{Hyperion Ltd.}{\emph{How to write a competitive proposal for Horizon 2020)}}
    \end{entrylist}
    
    \begin{entrylist}
    \entry
        {2014}{Coursera}{University of Michigan, US}{\emph{Model Thinking}}
    \end{entrylist}
    \begin{entrylist}
    \entry
        {2014}{Coursera}{Duke University,US}{\emph{Think Again: How to Reason and Argue}}
    \end{entrylist}
    
    \begin{entrylist}
    \entry
        {2013}{Fondazione Edmund Mach}{-}{\emph{Corso sicurezza dipendenti}}
    \end{entrylist}
    
    \begin{entrylist}
    \entry
        {2013}{Coursera}{Pennsylvania State University, US}{\emph{Maps and the Geospatial Revolution}}
    \end{entrylist}
    
    \begin{entrylist}
    \entry
        {2012}{OpenIT}{--}{\emph{GNU/Linux systems Administration}}
    \end{entrylist}
    
    \begin{entrylist}
    \entry
        {2011}{Italian AgroMeteorology Association}{--}{\emph{Introduction to agrometeorological models and cultural systems}}
    \end{entrylist}
    
    \begin{entrylist}
    \entry
        {2011}{Fondazione Edmund Mach}{Informatica Trentina}{\emph{Potenzialità di model builder in ambito SIAT}}
    \end{entrylist}   
    
    \begin{entrylist}
    \entry
        {2010}{Fondazione Edmund Mach}{Informatica Trentina}{\emph{Potenzialità di Spatial Analyst e 3D analyst in ambito SIAT}}
    \end{entrylist} 
    
    \begin{entrylist}
    \entry
        {2010}{Terradata Environmetrics}{Dipartimento di Scienze Ambientali, Università di Siena}{\emph{Statistical Analyses for Botanists and Ecologists}}
    \end{entrylist}
    
    \begin{entrylist}
    \entry
        {2010}{Fondazione Edmund Mach}{--}{\emph{Parametric and Nonparametric Statistics}}
    \end{entrylist}
    
    \begin{entrylist}
    \entry
        {2000}{Repubblica Italiana}{Commissariato del Governo per la Priovincia di Bolzano}{\emph{Attestato di bilinguismo (Diploma di istituto di istruzione secondaria di primo grado)}}
    \end{entrylist}
    
    \section{Informazioni personali}
    \begin{description}[style= unboxed, leftmargin= 6 pt, topsep= -3 pt, parsep= 3 pt, itemsep= 2pt]
        \item[Data di nascita:] 14 luglio 1976
        \item[Indirizzo:] Via Guardini, 63 - 38121 Trento (IT)
        \item[Telefono:] 0461~615395
        \item[Nazionalità:] italiana
        \item[Madrelingua:] italiano
    \end{description}
    
    
    \section{Pubblicazioni}
    \vspace{-0.75cm}
    \begin{refsection}
        \nocite{*}
        \newrefcontext[sorting=ydnt]
        \printbibliography[type=article, title={Riviste peer-review}, heading=subbibliography, notkeyword=divulgative]
    \end{refsection}
    
    \begin{refsection}
        \nocite{*}
        \newrefcontext[sorting=ydnt]
        \printbibliography[type=incollection, notkeyword=report, title={Capitoli di libro}, heading=subbibliography]
    \end{refsection}
        
    \begin{refsection}
        \nocite{*}
        \newrefcontext[sorting=ydnt]
        \printbibliography[type=inproceedings, title={Atti di convegno}, heading=subbibliography]
    \end{refsection}
    
     \begin{refsection}
        \nocite{*}
        \newrefcontext[sorting=ydnt]
        \printbibliography[type=incollection, keyword=report, title={Report tecnici}, heading=subbibliography]
    \end{refsection}       
    
     \begin{refsection}
        \nocite{*}
        \newrefcontext[sorting=ydnt]
        \printbibliography[type=article, title={Articoli divulgativi}, keyword=divulgative, heading=subbibliography]
    \end{refsection}    
    
    %https://www.researchgate.net/scientific-contributions/10134513_Paolo_Zatelli
    %https://iris.unitn.it/browse?type=author&authority=rp09824&authority_lang=it#.W4gPQRh9hhE
    
\end{document}
