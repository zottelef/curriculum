%% xelatex --interaction=nonstopmode curriculum_ita

\documentclass{curriculum}

\usepackage{kantlipsum}

\usepackage[italian]{babel}


\begin{document}
  \header{Fabio}{Zottele} {tecnologo sperimentatore di quarto livello}
  
    Mi sono laureato in ingegneria dell'ambiente e del territorio nel 2005 e nella prima parte del mio percorso professionale ho acquisito le competenze nella gestione di dati geografici per realizzare mappe digitali e per pubblicarle online.\\
    Dal 2006 lavoro presso la Fondazione Edmund Mach di San Michele all'Adige: attualmente sono assegnato al Centro di Trasferimento Tecnologico, Dipartimento Ambiente e agricoltura di montagna, Unità Agrometeorologia e sistemi informatici.\\
    Opero in autonomia in quattro ambiti:
    \begin{description}[style= unboxed, leftmargin= 0.5cm]
     \item[Analisi di dati nell'abito delle attività del mio gruppo di lavoro.] Utilizzo metodi statistici\textemdash sia frequentisti che  bayesiani\textemdash per l'interpolazione spaziotemporale delle misure effettuate dai sensori dalla rete agrometeorologica della Fondazione Mach. L'obiettivo è quello di controllare, correggere e validare le misure automatiche ed utilizzare le informazioni ottenute per l'analisi climatica, per il calcolo del rischio e del tempo di ritorno di eventi meteorologici che possono compromettere la produzione agricola, e per la realizzazione di mappe che coprono l'intero territorio provinciale.\\
     Inoltre, sviluppo un software per stimare il fabbisogno irriguo delle principali colture trentine in base all'evoluzione delle condizioni atmosferiche, della fisiologie della pianta e dello stato del terreno.
     %\\In questo ambito di lavoro ho supervisionato due collaboratoria progetto, un tirocinante e un tesista.    
     \item[Analisi di dati a supporto di altri gruppi della Fondazione Mach.] Metto a disposizione le mie conoscenze matematiche e statistiche in maniera trasversale ai Centri e ai Dipartimenti. Pianifico quali sono gli strumenti più adatti per l'analisi di dati agronomici ed ambientali e per l'interpretazione dei risultati. Per ottenere ciò mi avvalgo di software statistici e di tecnologie informative geografiche (GIS).
     %Nell'ambito di questa attività ho supervisionato due tesisti.
     \item[Studio dell'evoluzione dei paesaggi viticoli trentini] Dal 2010 indago come l'insieme di differenti fattori climatici, economici e sociali influenzano il paesaggio viticolo, guidandone i cambiamenti.\\In origine questo ambito di ricerca si è concentrato sullo sviluppo di strumenti informatici basati su GIS per la "misura" dei descrittori di un paesaggio al fine di mettere a confronto differenti territori viticoli europei (Val di Cembra in Italia e  Banyuls\textendash sur\textendash mer in Francia).
     Successivamente, per integrare "il fattore umano" nel descrivere l'evoluzione del territorio nello spazio e nel tempo ho sviluppato dei modelli multiagente. Da queste esperienze è nata una sintesi del lavoro che ha portato a definire il "Capitale Paesaggistico" ovvero uno strumento che consente non solo di confrontare territori distanti, ma anche di dare un valore tangibile al paesaggio come ulteriore strumento di valorizzazione delle produzioni vitivinicole. 
     %Nell'ambito di questa attività ho coordinato uno studente di master ed un tesista.
     \item[Droni e agricoltura di precisone] A partire dall' esperienze maturata sviluppando di sistemi di supporto alle decisioni, ho recentemente approfondito quali opportunità e quali benefici offra la meccatronica all'agricoltura trentina. In particolare, nell'ambito dell'utilizzo dei droni in agricoltura ho coordinato un progetto per l'utilizzo dei droni come strumento economico per il monitoraggio dell'evoluzione del paesaggio viticolo in forte pendenza e  coordino un progetto per coadiuvare il monitoraggio dei giallumi della vite per mezzo del drone.\\
     Sono referente per la Fondazione Mach sul Tavolo della Ricerca e della Sperimentazione, nato a seguito della firma del "protocollo di intesa per lo sviluppo della formazione aeronautica in Trentino".
     %Nell'ambito di questa attività ho coordinato un tirocinante ed due studenti: uno dell'Alta Formazione per Tecnico Superiore del Verde ed uno studente dell'Istituto Agrario.
    \end{description}
    
    Nell'ambito di queste attività ho coordinato il lavoro di due collaboratori a progetto laureati in ingegneria, uno studente di master, quattro tesisti universitari, un tirocinante laureato, due tirocinanti diplomati e uno studente di scuola superiore.
    
    \section{Istruzione}
    %Ricorda di inserire pubblicazoni recenti (anche quelle del curriculum per selezione c3a)
\end{document}
