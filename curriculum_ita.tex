%% xelatex --interaction=nonstopmode curriculum_ita

\documentclass{curriculum}

\usepackage{kantlipsum}

\usepackage[italian]{babel}


\begin{document}
  \header{Fabio}{Zottele} {tecnologo sperimentatore di quarto livello}
  
    Sono laureato in ingegneria dell'ambiente e del territorio nel 2005 e nella prima parte del mio percorso professionale ho acquisito le competenze per la gestione informatica di dati geografici, per la realizzazione di mappe digitali e per la loro pubblicazione online.\\
    Dal 2006 lavoro presso la Fondazione Edmund Mach di San Michele all'Adige: attualmente sono assegnato al Centro di Trasferimento tecnologico, Dipartimento Ambiente e agricoltura di montagna, Unità Agrometeorologia e sistemi informatici.\\
    Quattro sono gli ambiti di lavoro nei quali svolgo attività in autonomia:
    \begin{description}[style= unboxed, leftmargin= 0cm]
     \item[Analisi di dati nell'abito delle attività del mio gruppo di lavoro.] Utilizzo metodi statistici frequentisti e bayesiani per l'interpolazione spaziotemporale delle misure effettuate dalla rete agrometeorologica della Fondazione Mach. L'obiettivo è quello di correggere e validare le misure automatiche raccolte dai sensori ed utilizzaare quanto eleaborato nell'analisi climatica, nel calcolo del rischio e del tempo di ritorno di eventi meteorologici che possono compromettere la produzione agricola e nella realizzazione di mappe per l'intero territorio provinciale. Inoltre, utilizzo i risultati delle elaborazioni per alimentare e migliorare l'accuratezza del software che ho sviluppato per il calcolo del bilancio idrico. L'obiettivo è quello di fornire ai decisori adeguati sistemi di supporto alla gestione dell'acqua a scopo irriguo per tutto il territorio trentino. In questo ambito di attività ho di fatto supervisionato il lavoro di collaboratori, di tirocinanti e di un tesista;    
     \item[Analisi di dati a supporto di altri gruppi FEM.] Collaboro con altri gruppi di lavoro della Fondazione Mach per effettuare le analisi di dati di dati di tipo ambientale o agronomico dando supporto statistico e matematico. Nell'ambito di questa attività ho supervisionato tesisti.  d'uso intensivo di software statistici e di sistemi informativi geografici
     \item[studio dell'evoluzione dei paesaggi viticoli trentini] \kant[1];
     \item[droni e agricoltura di precsione] \kant[2];
    \end{description}
    

\end{document}
